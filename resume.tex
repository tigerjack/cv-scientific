\documentclass[
	a4paper, % Uncomment for A4 paper size (default is US letter)
	10pt, % Default font size, can use 10pt, 11pt or 12pt
]{tresume} % Use the resume class

\usepackage[english,printdayoff]{isodate}
\isodash{-}%
%

\usepackage{hyperref}
\usepackage[%
autocite     = plain,
backend      = biber,
doi          = true,
url          = true,
hyperref     = true,
language=english,
block=space,
maxcitenames=4
]{biblatex}
\addbibresource{papers.bib}
\addbibresource{reviews.bib}
%
\usepackage{tabularray}

\name{Simone Perriello} % Your name

\publictrue
\gdprfalse

\begin{document}
\ifpublic{%
    \begin{tblr}{colspec={X[2]X[4]X[2]X[4]},
        column{1,3} = {font=\itshape},
        width=\textwidth,
      }
      Email   & \href{mailto:simone.perriello@polimi.it}{simone.perriello@polimi.it}                    & Website  & \href{https://perriello.faculty.polimi.it}{https://perriello.faculty.polimi.it}
    \end{tblr}
  }
\else{%
    \begin{tblr}{colspec={X[2]X[4]X[2]X[4]},
        column{1,3} = {font=\itshape},
        width=\textwidth,
      }
      Date of birth    & {\origdate\printdayon\printdate{1989-07-25}} & Email   & \href{mailto:simone.perriello@polimi.it}{simone.perriello@polimi.it}            \\
      Citizenship      & Italian                                      & Phone   & \href{tel:(+39)0223999047}{(+39) 02 2399 9047}                                  \\
      Current location & Milan, Italy                                 & Website & \href{https://perriello.faculty.polimi.it}{https://perriello.faculty.polimi.it} 
    \end{tblr}
  }
\fi

%----------------------------------------------------------------------------------------
%	EDUCATION SECTION
%----------------------------------------------------------------------------------------
\begin{tSection}{Education}
  % PHD
  \begin{tSubsection}{Politecnico di Milano}{since \printdate{2019-11-01}}{Ph.D.\ candidate}{Milan}
  \item Thesis title: \emph{Quantum Circuits for Information Set Decoding: Quantum Cryptanalysis of Code-Based Cryptosystems}
  \item Advisors: Prof.\ \emph{Gerardo Pelosi}; Prof.\ \emph{Alessandro Barenghi}
  \end{tSubsection}
  % MASTER
  \begin{tSubsection}{Politecnico di Milano}{\printdate{2019-04-17}}{M.Sc.\ degree}{Milan}
  \item Thesis title: \emph{Design and developments of quantum circuits to solve the Information Set Decoding problem}
  \item Advisors: Prof.\ \emph{Gerardo Pelosi}; Prof.\ \emph{Alessandro
      Barenghi}
  \item Grade: 110/110
  \end{tSubsection}
  % 
  \begin{tSubsection}{IELTS}{\printdate{2016-02-01}}{English certificate}{}
  \item Thesis title: \textit{Quantum Computing Algorithms for Cryptography: design, validation and complexity assessment}
  \item Grade: 7.5/9 (equivalent to C1 of CEFR)
  \end{tSubsection}
  % 
  \begin{tSubsection}{Unisannio}{\printdate{2015-07-22}}{B.Sc.\ Degree}{Benevento}
  \item Thesis title: \textit{Un algoritmo per il social tagging di mashup}
  \item Advisor: Prof.\ \emph{Eugenio Zimeo}
  \item Grade: 110/110 cum laude
  \end{tSubsection}
\end{tSection}
%--------------------------------------------------------------------------------
%    RESEARCH
%-----------------------------------------------------------------------------------------------
\begin{tSection}{Research interests}
  My research journey has evolved from my B.Sc. thesis on \emph{recommender
    systems} to my current focus on \emph{quantum computing} and
  \emph{cryptography}. My current focus revolves around the meticulous design of
  quantum algorithms within the gate model, specifically aimed at addressing
  challenges in code-based cryptosystems.

  During my Bachelor's studies, I specialized in designing a recommendation
  system tailored for enhancing social tagging of mashups \textemdash{} web
  applications that integrate data and functionalities from diverse sources to
  enhance the overall user experience. %
  The principal goal of this project was to enhance the core of the mashup
  tagging process by optimizing the recommender system. This optimization was
  achieved through systematic social data merging and the discernment of
  potential user relationships.

  % The primary objective of this project was to improve the recommender system
  % at the core of the mashup tagging process through the social data merging,
  % the uncovering of potential user relationship.
  %
  % by advancing social data merging and
  % uncovering potential user relations. The emphasis was on refining the
  % recommender system
  %
  % tagging process within Web 2.0 systems, ensuring a more effective and
  % user-centric approach to suggesting tags based on enriched social data and
  % identified user relationships.
  %
  % Employing no-SQL databases, the project aimed to merge social data,
  % identify potential user relations, and spotlight community leaders. This
  % initiative significantly contributed to the improvement of the tagging
  % process at the core of Web 2.0 systems.

  Building on this foundation, my Master's program became a platform for a
  self-guided exploration of quantum computing. This journey culminated in my
  thesis, during which I developed a quantum adaptation of the \emph{Information
    Set Decoding (ISD)} strategy, the most efficient kind of attack against
  cryptosystems based on linear codes. The implementation of those attacks was
  based on IBM's open source Qiskit framework, to which I also contributed
  several patches.

  During my internship at Atos, I extended my research by enhancing quantum
  algorithm simulations for Noisy Intermediate-Scale Quantum (NISQ)
  architectures. I created a versatile quantum simulation library capable of
  simulating systems with hundreds of qubits, targeted for the Atos'
  \emph{Quantum Learning Machine (QLM)} environment. The library was extensively
  used to replicate state of the art experimental results related to the
  challenging \emph{barren plateau problem} in quantum neural networks.

  My Ph.D.\ research centred on \emph{quantum cryptanalysis} of post-quantum
  cryptography. I proposed the first complete design of quantum circuits
  tailored to attack the hardness assumptions in code-based cryptography,
  evaluating the computational complexity of attacking all the code-based
  cryptosystems under international scrutiny.
  %
  Comprehensive assessments and comparisons, which considered both theoretical
  and practical implementations for quantum ISD introduced in the years
  following my initial work, confirmed the substantial advantage of my
  contribution, with performance surpassing other approaches by a significant
  margin, ranging from $2^{19}$ to $2^{30}$.

  During this process, I also designed a range of practical quantum circuits
  that can be of independent interests --- to sort bitstrings, to permute matrix
  columns, to perform Gauss-Jordan Elimination on a matrix, and to check the
  weight of a given bitstring.

\end{tSection}
\clearpage
%----------------------------------------------------------------------------------------
%	WORK EXPERIENCE SECTION
%----------------------------------------------------------------------------------------
\begin{tSection}{Work Experience}
% R&D Bull SAS
  \begin{tSubsection}{Atos: Bull SAS R\&D Labs}{\origdate\daterange{2020-02-01}{2020-07-31}}{Quantum computing researcher}{Les Clayes-sous-Bois}
  \item \emph{Supervisors}: \emph{Bertrand Marchand} and \emph{Cyril Allouche}
  \item Implemented novel simulation strategies for quantum circuits targeting
    NISQ architecture.
  \item Explored the \emph{barren plateau problem} in quantum neural network.
    % \item Tested and implemented algorithms on the \textit{Atos Quantum Learning Machine} simulator.
  \end{tSubsection}

  % ATOS QUANTUM
  \begin{tSubsection}{Atos: HPC \& Quantum team}{\origdate\daterange{2019-07-01}{2020-01-31}}{Quantum computing researcher}{Milan}
  \item Configured hardware/software stack of the \textit{Atos QLM} appliance.
  \item Implemented well-known quantum algorithms on the \textit{Atos QLM} framework.
  \item Lectured external customers on the \textit{Atos QLM} framework.
\end{tSubsection}
\end{tSection}
%----------------------------------------------------------------------------------------
%	TEACHING
%----------------------------------------------------------------------------------------
\begin{tSection}{Teaching Experience}
  \begin{tSubsection}{Teaching assistant at Politecnico di Milano}{\daterange{2022-02-23}{2022-03-30}}{Introduction
    to Quantum Computing (Ph.D.\ course)}{Prof.\ Gerardo Pelosi, Prof.\ Alessandro Barenghi}
  \item Presentation of the Atos myQLM and QLM frameworks.
  \item Showcase code implementation of renowned quantum algorithms using QLM
    framework.
  \end{tSubsection}
  \begin{tSubsection}{Teaching assistant at Politecnico di Milano}{\daterange{2020-11-01}{2024-01-31}}{Computer Architectures and Operating Systems}{Prof.\ Gerardo Pelosi}
  \item Exercise lectures: Linux Operating Systems.
  \item Topics addressed (partial): parallel programming (processes, threads),
    task scheduler, system calls and interrupt routines, memory management, file
    systems and I/O.
  \end{tSubsection}
  \begin{tSubsection}{Teaching assistant at Politecnico di Milano}{\daterange{2022-11-01}{2024-01-31}}{Computer Architectures and
      Operating Systems}{Prof.\ Cristina Silvano }
  \item Exercise lectures: Linux Operating Systems.
  \item Topics addressed (partial): parallel programming (processes, threads),
    task scheduler, system calls and interrupt routines, memory management, file
    systems and I/O.
  \end{tSubsection}
  \begin{tSubsection}{Teaching assistant at Politecnico di Milano}{\daterange{2023-11-01}{2024-01-31}}{Computer Architectures and
      Operating Systems}{Prof.\ Federico Terraneo}
  \item Exercise lectures: Linux Operating Systems.
  \item Topics addressed (partial): parallel programming (processes, threads),
    task scheduler, system calls and interrupt routines, memory management, file
    systems and I/O.
  \end{tSubsection}
  \begin{tSubsection}{Teaching assistant at Politecnico di Milano}{\daterange{2021-11-01}{2022-06-30}}{Informatica (per Aerospaziali)}{Prof.\ Gerardo Pelosi}
  \item Exercise lectures: computer science for Aerospace Engineering.
  \item Topics addressed (partial): Boolean logic and basics of C programming.
  \end{tSubsection}
  \begin{tSubsection}{Teaching tutor at Politecnico di Milano}{\daterange{2018-11-24}{2019-01-17}}{Informatica (per Ambientali)}{Prof.\ Andrea Bonarini}
\item Theory lectures and laboratory exercises on the C programming language.
% https://www11.ceda.polimi.it/schedaincarico/schedaincarico/controller/scheda_pubblica/SchedaPublic.do?&evn_default=evento&c_classe=691950&polij_device_category=DESKTOP&__pj0=0&__pj1=ee13b0b6c0f983f39edcd5af2ad25bd3
\end{tSubsection}
\begin{tSubsection}{Teaching tutor at Politecnico di Milano}{\daterange{2016-11-31}{2017-01-31}}{Computer
    Architectures and Operating Systems}{Prof.\ Anna Maria Antola}
\item Exercise lectures: Linux Operating Systems.
\item Topics addressed (partial): parallel programming (processes, threads),
    task scheduler, system calls and interrupt routines, memory management, file
    systems and I/O.
% https://www11.ceda.polimi.it/schedaincarico/schedaincarico/controller/scheda_pubblica/SchedaPublic.do?&evn_default=evento&c_classe=667442&polij_device_category=DESKTOP&__pj0=0&__pj1=f7cded5163ff00a70c340961c9ce685c
\end{tSubsection}
\end{tSection}
\clearpage
\begin{tSection}{List of publications}
  \begin{tSubPublications}{Refereed International Journals}{J}
  \item \fullcite{perriello2023ImprovingEfficiencyQuantum}
  \end{tSubPublications}
  \begin{tSubPublications}{Refereed International Conferences}{C}
  \item \fullcite{lancellotti2023DesignQuantumWalkConf}
  \item \fullcite{perriello2024QuantumCircuitDesign}
  \item \fullcite{DBLP:conf/qce/PerrielloBP21}
  \item \fullcite{DBLP:conf/securecomm/PerrielloBP21}
  \end{tSubPublications}
  \begin{tSubPublications}{Non-Refereed}{N}
  \item Poster at \emph{International Summer School on Advanced
      Computer Architecture and Compilation for High-performance Embedded
      Systems} with title \emph{A Quantum Circuit to Speed-up the Cryptanalysis
      of Code-based Cryptosystems}.
  \end{tSubPublications}
\end{tSection}
%
\begin{tSection}{Reviewer for International Journals and Conferences}
  \begin{tSubsection}{2024}{}{}{}
  \item \fullcite{DBLP:conf/qce/2024}
  \item \fullcite{tifs}, Impact Factor: 6.8, SCImago Journal Rank (SJR) 2023: 2.89 (Q1)
  \item \fullcite{tetc}, Impact Factor: 5.9, SCImago Journal Rank (SJR) 2023:
    1.57 (Q1)
  \end{tSubsection}
  \begin{tSubsection}{2023}{}{}{}
  \item \fullcite{DBLP:conf/icissp/2023}
  \item \fullcite{DBLP:conf/acisp/2023}
  \end{tSubsection}
  \begin{tSubsection}{2021}{}{}{}
  \item \fullcite{DBLP:conf/iccad/2021}
  \end{tSubsection}
\end{tSection}
% \begin{tSection}{Additional scientific activities}
%   \begin{itemize}
%   \end{itemize}
% \end{tSection}
%
\begin{tSection}{Program Committee member}
  \begin{tSubsection}{2024}{}{}{}
  % \begin{tSubPublications}{Program Committee member}{PC}
  \item \emph{IEEE International Conference on Quantum Computing and
      Engineering, QCE 2024, Montréal, Québec, Canada, September 15–20, 2024. IEEE
      2024, ISBN XXX}
  % \end{tSubPublications}
\end{tSubsection}
\end{tSection}
%
\begin{tSection}{Awards and Recognition}
  \begin{itemize}
  \item[2024] Grant winner for \emph{61st ACM/IEEE Design Automation
      Conference, DAC 2024, San Francisco, CA, USA, July 23-27,
      2024}.
  \item[2021] Grant winner for \emph{International Summer School on Advanced
      Computer Architecture and Compilation for High-performance Embedded
      Systems}.
  \end{itemize}
\end{tSection}
\vfill
\ifpublic{%
    \ifgdpr{%
    \begin{itemize}
\item
I authorize the processing of data pursuant to GDPR 2016/679 of 27 April 2016 (European Regulation concerning the protection of individuals with regard to the processing of personal data).
\item
    I authorize the publication of the Curriculum Vitae on the website of the Politecnico di Milano (Section Transparent Administration) in compliance with Legislative Decree no. 33 of March 14, 2013 and any subsequent amendments and additions
\end{itemize}
    \begin{itemize}
\item Autorizzo al trattamento dati ai sensi del GDPR 2016/679 del 27 aprile 2016 (Regolamento Europeo relativo alla protezione delle persone fisiche per quanto riguarda il trattamento dei dati personali).
\item Autorizzo la pubblicazione del Curriculum Vitae sul sito istituzionale del Politecnico di Milano (sez. Amministrazione Trasparente) in ottemperanza al D. Lgs n. 33 del 14 marzo 2013 (e s.m.i.).
\end{itemize}
  }
  \fi
}
\fi
\end{document}
