\documentclass[
	a4paper, % Uncomment for A4 paper size (default is US letter)
	10pt, % Default font size, can use 10pt, 11pt or 12pt
]{tresume} % Use the resume class

\usepackage[english,printdayoff]{isodate}
\isodash{-}%
%

\usepackage{hyperref}
\usepackage[%
autocite     = plain,
backend      = biber,
doi          = true,
url          = true,
hyperref     = true,
language=english,
block=space,
maxcitenames=4
]{biblatex}
\addbibresource{papers.bib}
\addbibresource{reviews.bib}
\addbibresource{service.bib}
%
\usepackage{tabularray}

\name{Simone Perriello}

% NOTE change these flags to have different outputs
\publictrue
\gdprfalse

\begin{document}
\ifpublic{%
    \begin{tblr}{colspec={X[2]X[4]X[2]X[4]},
        column{1,3} = {font=\itshape},
        width=\textwidth,
      }
      Email   & \href{mailto:simone.perriello@polimi.it}{simone.perriello@polimi.it}                 & Website  & \href{https://perriello.faculty.polimi.it}{https://perriello.faculty.polimi.it}\\
      ORCID   &
      \href{https://orcid.org/0000-0001-9656-7252}{0000-0001-9656-7252}
      & Lab Website  & \href{https://www.heaplab.deib.polimi.it}{https://www.heaplab.deib.polimi.it}
    \end{tblr}
  }
\else{%
    \begin{tblr}{colspec={X[2]X[4]X[2]X[4]},
        column{1,3} = {font=\itshape},
        width=\textwidth,
      }
      Date of birth    & {\origdate\printdayon\printdate{1989-07-25}} & Email   & \href{mailto:simone.perriello@polimi.it}{simone.perriello@polimi.it}            \\
      Citizenship      & Italian                                      & Phone   & \href{tel:(+39)0223999047}{(+39) 02 2399 9047}                                  \\
      Current location & Milan, Italy                                 & Website & \href{https://perriello.faculty.polimi.it}{https://perriello.faculty.polimi.it} 
    \end{tblr}
  }
\fi

%----------------------------------------------------------------------------------------
%	ACADEMIC EXPERIENCE SECTION
%----------------------------------------------------------------------------------------
\begin{tSection}{Academic Experience}
  % PostDoc
  \begin{tSubsection}{Politecnico di Milano}{since \printdate{2024-05-16}}{Postdoctoral Researcher}{Milan}
  \item Project 1: Quantum Cryptanalysis of Symmetric and Asymmetric Cryptosystems
  \item Project 2: Quantum Acceleration of Clique Problems
  \end{tSubsection}
  % PhD
  \begin{tSubsection}{Politecnico di Milano}{\printdate{2024-05-15}}{Ph.D.}{Milan}
  \item Thesis title: \emph{Quantum Circuits for Information Set Decoding: Quantum Cryptanalysis of Code-Based Cryptosystems}
  \item Advisors: Prof.\ \emph{Gerardo Pelosi}; Prof.\ \emph{Alessandro Barenghi}
  \end{tSubsection}
\end{tSection}
%----------------------------------------------------------------------------------------
%	WORK EXPERIENCE SECTION
%----------------------------------------------------------------------------------------
\begin{tSection}{Work Experience}
% R&D Bull SAS
  \begin{tSubsection}{Atos: Bull SAS R\&D Labs}{\origdate\daterange{2020-02-01}{2020-07-31}}{Quantum computing researcher}{Les Clayes-sous-Bois}
  \item \emph{Supervisors}: \emph{Bertrand Marchand} and \emph{Cyril Allouche}
  \item Implemented novel simulation strategies for quantum circuits targeting
    NISQ architecture.
  \item Explored the \emph{barren plateau problem} in quantum neural network.
    % \item Tested and implemented algorithms on the \textit{Atos Quantum Learning Machine} simulator.
  \end{tSubsection}

  % ATOS QUANTUM
  \begin{tSubsection}{Atos: HPC \& Quantum team}{\origdate\daterange{2019-07-01}{2020-01-31}}{Quantum computing researcher}{Milan}
  \item Configured hardware/software stack of the \textit{Atos QLM} appliance.
  \item Implemented well-known quantum algorithms on the \textit{Atos QLM} framework.
  \item Lectured external customers on the \textit{Atos QLM} framework.
\end{tSubsection}
\end{tSection}

%----------------------------------------------------------------------------------------
%	EDUCATION SECTION
%----------------------------------------------------------------------------------------
\begin{tSection}{Education}
  % MASTER
  \begin{tSubsection}{Politecnico di Milano}{\printdate{2019-04-17}}{M.Sc.\ degree}{Milan}
  \item Thesis title: \emph{Design and developments of quantum circuits to solve the Information Set Decoding problem}
  \item Advisors: Prof.\ \emph{Gerardo Pelosi}; Prof.\ \emph{Alessandro
      Barenghi}
  \item Grade: 110/110
  \end{tSubsection}
  % 
  % \begin{tSubsection}{IELTS}{\printdate{2016-02-01}}{English certificate}{}
  % \item Thesis title: \textit{Quantum Computing Algorithms for Cryptography: design, validation and complexity assessment}
  % \item Grade: 7.5/9 (equivalent to C1 of CEFR)
  % \end{tSubsection}
  % 
  \begin{tSubsection}{Unisannio}{\printdate{2015-07-22}}{B.Sc.\ Degree}{Benevento}
  \item Thesis title: \textit{Un algoritmo per il social tagging di mashup}
  \item Advisor: Prof.\ \emph{Eugenio Zimeo}
  \item Grade: 110/110 cum laude
  \end{tSubsection}
\end{tSection}
%--------------------------------------------------------------------------------
%    RESEARCH
%-----------------------------------------------------------------------------------------------
\begin{tSection}{Research interests}
    My research focuses on \emph{quantum computing} and \emph{quantum cryptanalysis}, with a particular emphasis on code-based cryptography. During my Master's thesis, I designed quantum circuits adapting the \emph{Information Set Decoding (ISD)} strategy, the most efficient known attack on code-based cryptosystems, implementing and benchmarking these techniques using IBM’s Qiskit framework. I expanded this work during my Ph.D. by designing optimized quantum circuits to evaluate the complexity of attacking all major code-based cryptographic schemes under international evaluation. The research led to significant reductions in attack complexity, improving efficiency by factors ranging from $2^{19}$ to $2^{30}$ compared to previous approaches.

During my time at Atos' R\&D laboratories, I developed a quantum simulation library for Noisy Intermediate-Scale Quantum (NISQ) architectures, enabling large-scale simulations on the \emph{Quantum Learning Machine (QLM)}. This tool was instrumental in reproducing state-of-the-art results on quantum neural networks, particularly in studying the \emph{barren plateau problem}.

As a Postdoctoral researcher, I continue to explore the intersection of quantum algorithm design and cryptographic security, contributing to the study of post-quantum cryptographic resilience. My current research also extends to the design of quantum algorithms for graph-related problems, such as clique detection, and optimizing input state preparation techniques.

\end{tSection}
%----------------------------------------------------------------------------------------
%	TEACHING
%----------------------------------------------------------------------------------------
\begin{tSection}{Teaching Experience}
  \begin{tSubsection}{Teaching assistant at Politecnico di Milano}{\daterange{2023-11-01}{2025-01-31}}{Computer Architectures and
      Operating Systems}{Prof.\ Federico Terraneo}
  \item Exercise lectures: Linux Operating Systems.
  \item Topics addressed (partial): parallel programming (processes, threads),
    task scheduler, system calls and interrupt routines, memory management, file
    systems and I/O.
  \end{tSubsection}
  \begin{tSubsection}{Teaching assistant at Politecnico di Milano}{\daterange{2022-11-01}{2025-01-31}}{Computer Architectures and
      Operating Systems}{Prof.\ Cristina Silvano }
  \item Exercise lectures: Linux Operating Systems.
  \item Topics addressed (partial): parallel programming (processes, threads),
    task scheduler, system calls and interrupt routines, memory management, file
    systems and I/O.
  \end{tSubsection}
  \begin{tSubsection}{Teaching assistant at Politecnico di Milano}{\daterange{2022-02-23}{2022-03-30}}{Introduction
    to Quantum Computing (Ph.D.\ course)}{Prof.\ Gerardo Pelosi, Prof.\ Alessandro Barenghi}
  \item Presentation of the Atos myQLM and QLM frameworks.
  \item Showcase code implementation of renowned quantum algorithms using QLM
    framework.
  \end{tSubsection}
  \begin{tSubsection}{Teaching assistant at Politecnico di Milano}{\daterange{2020-11-01}{2025-01-31}}{Computer Architectures and Operating Systems}{Prof.\ Gerardo Pelosi}
  \item Exercise lectures: Linux Operating Systems.
  \item Topics addressed (partial): parallel programming (processes, threads),
    task scheduler, system calls and interrupt routines, memory management, file
    systems and I/O.
  \end{tSubsection}
  \begin{tSubsection}{Teaching assistant at Politecnico di Milano}{\daterange{2021-11-01}{2022-06-30}}{Informatica (per Aerospaziali)}{Prof.\ Gerardo Pelosi}
  \item Exercise lectures: computer science for Aerospace Engineering.
  \item Topics addressed (partial): Boolean logic and basics of C programming.
  \end{tSubsection}
  \begin{tSubsection}{Teaching tutor at Politecnico di Milano}{\daterange{2018-11-24}{2019-01-17}}{Informatica (per Ambientali)}{Prof.\ Andrea Bonarini}
\item Theory lectures and laboratory exercises on the C programming language.
% https://www11.ceda.polimi.it/schedaincarico/schedaincarico/controller/scheda_pubblica/SchedaPublic.do?&evn_default=evento&c_classe=691950&polij_device_category=DESKTOP&__pj0=0&__pj1=ee13b0b6c0f983f39edcd5af2ad25bd3
\end{tSubsection}
\begin{tSubsection}{Teaching tutor at Politecnico di Milano}{\daterange{2016-11-31}{2017-01-31}}{Computer
    Architectures and Operating Systems}{Prof.\ Anna Maria Antola}
\item Exercise lectures: Linux Operating Systems.
\item Topics addressed (partial): parallel programming (processes, threads),
    task scheduler, system calls and interrupt routines, memory management, file
    systems and I/O.
% https://www11.ceda.polimi.it/schedaincarico/schedaincarico/controller/scheda_pubblica/SchedaPublic.do?&evn_default=evento&c_classe=667442&polij_device_category=DESKTOP&__pj0=0&__pj1=f7cded5163ff00a70c340961c9ce685c
\end{tSubsection}
\end{tSection}
\begin{tSection}{List of publications}
  \begin{tSubPublications}{Refereed International Journals}{J}
  \item \fullcite{perriello2023ImprovingEfficiencyQuantum}
  \end{tSubPublications}
  \begin{tSubPublications}{Refereed International Conferences}{C}
  \item \fullcite{DBLP:sp_cf25_tmp}
  \item \fullcite{perriello2024AQuantumCircuitToExecuteAK}
  \item \fullcite{lancellotti2023DesignQuantumWalkConf}
  \item \fullcite{perriello2024QuantumCircuitDesign}
  \item \fullcite{DBLP:conf/qce/PerrielloBP21}
  \item \fullcite{DBLP:conf/securecomm/PerrielloBP21}
  \end{tSubPublications}
  \begin{tSubPublications}{Non-Refereed}{N}
  \item Poster at \emph{International Summer School on Advanced
      Computer Architecture and Compilation for High-performance Embedded
      Systems} with title \emph{A Quantum Circuit to Speed-up the Cryptanalysis
      of Code-based Cryptosystems}.
  \end{tSubPublications}
\end{tSection}
%
\begin{tSection}{Reviewer for International Journals and Conferences}
  \begin{tSubsection}{2025}{}{}{}
  \item \fullcite{qce25}
  \item \fullcite{force25}
  \item \fullcite{fgcs25}, Impact Factor: 6.2, SCImago Journal Rank (SJR) 2023: 1.95 (Q1)
  \item \fullcite{cf25}
  \end{tSubsection}
  \begin{tSubsection}{2024}{}{}{}
  \item \fullcite{qce24}
  \item \fullcite{tifs24}, Impact Factor: 6.8, SCImago Journal Rank (SJR) 2023: 2.89 (Q1)
  \item \fullcite{tetc24}, Impact Factor: 5.9, SCImago Journal Rank (SJR) 2023: 1.57 (Q1)
  \end{tSubsection}
  \begin{tSubsection}{2023}{}{}{}
  \item \fullcite{DBLP:conf/icissp/2023}
  \item \fullcite{DBLP:conf/acisp/2023}
  \end{tSubsection}
  \begin{tSubsection}{2021}{}{}{}
  \item \fullcite{DBLP:conf/iccad/2021}
  \end{tSubsection}
\end{tSection}
% \begin{tSection}{Additional scientific activities}
%   \begin{itemize}
%   \end{itemize}
% \end{tSection}
%
\begin{tSection}{Program Committee member}
  \begin{tSubsection}{2025}{}{}{}
    \item \fullcite{service_qce25_tmp}
    \item \fullcite{service_cf25_tmp}
    \item \fullcite{service_force25_tmp}
  \end{tSubsection}
  \begin{tSubsection}{2024}{}{}{}
    \item \fullcite{service_qce24}
\end{tSubsection}
\end{tSection}
%
\begin{tSection}{Awards and Recognition}
  \begin{itemize}
  \item[2024] HiPEAC Paper Award, \emph{European Network of Excellence on High
      Performance and Embedded Architecture and Compilation}
  \item[2024] Grant winner for \emph{61st ACM/IEEE Design Automation
      Conference, DAC 2024, San Francisco, CA, USA, July 23-27,
      2024}.
  \item[2021] Grant winner for \emph{International Summer School on Advanced
      Computer Architecture and Compilation for High-performance Embedded
      Systems}.
  \end{itemize}
\end{tSection}

\begin{tSection}{Other academic achievements, honors, and activities}
  \begin{tSubsection}{2024}{}{}{}
    \item Session chair for the session titled \emph{Application for Data
        Analysis} at~\fullcite{service_qce24}
  \end{tSubsection}

\end{tSection}
\vfill
\ifpublic{%
    \ifgdpr{%
    \begin{itemize}
\item
I authorize the processing of data pursuant to GDPR 2016/679 of 27 April 2016 (European Regulation concerning the protection of individuals with regard to the processing of personal data).
\item
    I authorize the publication of the Curriculum Vitae on the website of the Politecnico di Milano (Section Transparent Administration) in compliance with Legislative Decree no. 33 of March 14, 2013 and any subsequent amendments and additions
\end{itemize}
    \begin{itemize}
\item Autorizzo al trattamento dati ai sensi del GDPR 2016/679 del 27 aprile 2016 (Regolamento Europeo relativo alla protezione delle persone fisiche per quanto riguarda il trattamento dei dati personali).
\item Autorizzo la pubblicazione del Curriculum Vitae sul sito istituzionale del Politecnico di Milano (sez. Amministrazione Trasparente) in ottemperanza al D. Lgs n. 33 del 14 marzo 2013 (e s.m.i.).
\end{itemize}
  }
  \fi
}
\fi
\end{document}
