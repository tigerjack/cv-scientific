%%%%%%%%%%%%%%%%%%%%%%%%%%%%%%%%%%%%%%%%%
% Medium Length Professional CV
% LaTeX Template
% Version 2.0 (8/5/13)
%
% This template has been downloaded from:
% http://www.LaTeXTemplates.com
%
% Important note:
% This template requires the resume.cls file to be in the same directory as the
% .tex file. The resume.cls file provides the resume style used for structuring the
% document.
%
%%%%%%%%%%%%%%%%%%%%%%%%%%%%%%%%%%%%%%%%%

%----------------------------------------------------------------------------------------
%	PACKAGES AND OTHER DOCUMENT CONFIGURATIONS
%----------------------------------------------------------------------------------------
\documentclass[
	a4paper, % Uncomment for A4 paper size (default is US letter)
	11pt, % Default font size, can use 10pt, 11pt or 12pt
]{resume} % Use the resume class

\usepackage{ebgaramond} % Use the EB Garamond font
% \usepackage[left=0.7in,top=0.6in,right=0.7in,bottom=0.6in]{geometry} % Document margins
\usepackage{fancyhdr}
\usepackage[english,printdayoff]{isodate}
\isodash{-}%
\newcommand{\tab}[1]{\hspace{.2667\textwidth}\rlap{#1}}
\newcommand{\itab}[1]{\hspace{0em}\rlap{#1}}
\setlength{\tabcolsep}{18pt}
\newenvironment{mytabular}
  {\trivlist\item
    \tabular{ll}}
  {\endtabular\endtrivlist}
\newcommand{\row}[2]{\textit{#1}&#2\\}
\usepackage[%
autocite     = plain,
backend      = biber,
doi          = true,
url          = true,
giveninits   = true,
hyperref     = true,
maxbibnames  = 99,
maxcitenames = 99,
sortcites    = true,
sortlocale   = en-GB,
% style        = alphabetic,
style        = iso-alphabetic,
language=english,
% next two for bibliography overfull
% https://tex.stackexchange.com/a/442317/80949 block=ragged,
block=space,
]{biblatex}
\addbibresource{papers.bib}
\addbibresource{reviews.bib}

\usepackage{enumitem}

\newif\ifpublic{}
% by default it is set to false
\publicfalse{}

% \makeatletter
% \def\myrev{Next Review Date:}
% \def\ps@lastpage{%
%   \let\@oddhead\@empty\let\@evenhead\@empty%
%   \def\@oddfoot{{\slshape\myrev}\hfil\thepage}%
%   \def\@evenfoot{{\slshape\myrev}\hfil\thepage}%
%   }
% \makeatother


\name{Simone Perriello} % Your name
\begin{document}
\ifpublic{%
    \begin{mytabular}
      \row{Academic email}{simone.perriello@polimi.it}
    \end{mytabular}%
  }
\else{%
    \begin{mytabular}
          %   \row{Date of Birth}{July, 25th 1989}
          %   \row{Citizenship}{Italian}
      \row{Academic email}{simone.perriello@polimi.it}
          %   \row{Personal email}{simone.perriello@protonmail.com}
      \row{Personal email}{sperriello@proton.me}
      % \row{Github page}{github.com/tigerjack}
      \row{Phone number}{(+39) 02 2399 9047}
      \row{Website}{https://perriello.faculty.polimi.it}
    \end{mytabular}%
  }
\fi

%----------------------------------------------------------------------------------------
%	EDUCATION SECTION
%----------------------------------------------------------------------------------------
\begin{rSection}{Education}
% PHD
\textbf{Ph.D. Candidate} \hfill \textit{since November 2019}
\\ Enrolled in Ph.D program in Information Technology at \emph{Politecnico di Milano}
\\ Thesis title: \textit{Quantum Computing Algorithms for Cryptography: design, validation and complexity assessment}
\\ Advisor Prof.\ \textit{Gerardo Pelosi}; Co-Advisor Prof.\ \textit{Alessandro Barenghi}
\\
\\
% MASTER
\textbf{M.Sc.\ degree} \hfill \textit{April 2019}
\\ Master of Science in Computer Science and Engineering at \emph{Politecnico di Milano}
\\ Thesis title: \textit{Design and developments of quantum circuits to solve the Information Set Decoding problem}
\\ Advisor Prof.\ \textit{Gerardo Pelosi}; Co-Advisor Prof.\ \textit{Alessandro Barenghi}; Grade: 110/110
\\
\\
% IELTS
\textbf{IELTS} \hfill \textit{February 2016}
\\ Grade 7.5/9 (equivalent to C1 of the CEFR)
\\
\end{rSection}
%--------------------------------------------------------------------------------
%    RESEARCH
%-----------------------------------------------------------------------------------------------
\begin{rSection}{Research interests}
  My research spans the domains of \emph{quantum computing} and
  \emph{cryptography}, with a primary focus on designing quantum algorithms
  based on the gate model to attack code-based cryptosystems.

  During my Master's program, I embarked on a self-guided exploration of quantum
  computing. This journey culminated in my thesis, during which I developed a
  quantum adaptation of the \emph{Information Set Decoding (ISD)} strategy, the
  most efficient kind of attack against cryptosystems based on linear codes. The
  implementation of those attacks was based on IBM's open source Qiskit
  framework, to which I also contributed several patches.

  During my internship at Atos, I extended my research by enhancing quantum
  algorithm simulations for Noisy Intermediate-Scale Quantum (NISQ)
  architectures. I created a versatile quantum simulation library capable of
  simulating systems with hundreds of qubits, targeted for the Atos' Quantum
  Learning Machine environment. The library was extensively used to replicate
  state of the art experimental results related to the challenging \emph{barren
    plateau problem} in quantum neural networks.

  My Ph.D.\ research centered on \emph{quantum cryptanalysis} of Post-Quantum
  Cryptography (PQC). I proposed the first complete design of quantum circuits
  tailored to attack the hardness assumptions in code-based crpytography,
  evaluating the computational complexity of attacking all the code-based
  cryptosystems under international scrutiny.
  %
  Comprehensive assessments and comparisons, which considered both theoretical
  and practical implementations for quantum ISD introduced in the years
  following my initial work, confirmed the substantial advantage of my
  contribution, with performance surpassing other approaches by a significant
  margin, ranging from $2^{19}$ to $2^{30}$.

  During this process, I also designed a range of practical quantum circuits
  that can be of independent interests --- to sort bitstrings, to permute matrix
  columns, to perform Gauss-Jordan Elimination on a matrix, and to check the
  weight of a given bitstring.

\end{rSection}
\clearpage
%----------------------------------------------------------------------------------------
%	TECHNICAL STRENGTHS SECTION
%----------------------------------------------------------------------------------------
% \begin{rSection}{Technical Strengths}
% % PYTHON
% % QUANTUM
% \begin{tabular}{ @{} >{\bfseries}l @{\hspace{6ex}} l }
% Modeling and Analysis \ & AutoCad, Revit, StaadPro \\
% Software \& Tools & MS Office, Latex \\
% \end{tabular}
%
% \end{rSection}
%----------------------------------------------------------------------------------------
%	WORK EXPERIENCE SECTION
%----------------------------------------------------------------------------------------
% NOTE Dates are in US format
\begin{rSection}{Work Experience}
% R&D Bull SAS
% \begin{rSubsection}{Atos: Bull SAS R\&D Labs}{February 2020--July 2020}{Quantum computing researcher}{Les Clayes-sous-Bois}
  \begin{rSubsection}{Atos: Bull SAS R\&D Labs}{\origdate\daterange{2020-02-01}{2020-07-31}}{Quantum computing researcher}{Les Clayes-sous-Bois}
  \item \emph{Supervisors}: Bertrand Marchand and Cyril Allouche.
  \item Implemented novel simulation strategies for quantum circuits targeting
    NISQ architecture
  \item Explored the \emph{barren plateau problem} in quantum neural network.
    % \item Tested and implemented algorithms on the \textit{Atos Quantum Learning Machine} simulator.
  \end{rSubsection}

  % ATOS QUANTUM
  % \begin{rSubsection}{Atos: HPC \& Quantum team}{July 2019--January 2020}{Quantum computing researcher}{Milano}
  \begin{rSubsection}{Atos: HPC \& Quantum team}{\origdate\daterange{2019-07-01}{2020-01-31}}{Quantum computing researcher}{Milano}
  \item Configured hardware/software stack of the \textit{Atos Quantum Learning Machine} appliance.
  \item Implemented well-known quantum algorithms on the \textit{Atos Quantum
      Learning Machine} appliance.
\item Lectured external customers on the \textit{Atos Quantum Learning Machine} software stack.
% \item Taken internal courses on the advanced usage of the \textit{Atos Quantum Learning Machine}.
\end{rSubsection}

\end{rSection}
%----------------------------------------------------------------------------------------
%	TEACHING
%----------------------------------------------------------------------------------------
\begin{rSection}{Teaching Experience}

  \begin{rSubsection}{Teaching assistant}{2020-21;\ 21-22;\ 22-23;\ 23-24}{Computer Architectures and Operating Systems}{Prof.\ Gerardo Pelosi}
  \item Exercise lectures: Linux Operating Systems.
  \item Topics addressed (partial): parallel programming (processes, threads),
    task scheduler, system calls and interrupt routines, memory management, file
    systems and I/O.
  \end{rSubsection}
  \begin{rSubsection}{Teaching assistant}{2021-22;\ 22-23;\ 23-24}{Computer Architectures and
      Operating Systems}{Prof.ssa Cristina Silvano }
  \item Exercise lectures: Linux Operating Systems.
  \item Topics addressed (partial): parallel programming (processes, threads),
    task scheduler, system calls and interrupt routines, memory management, file
    systems and I/O.
  \end{rSubsection}
  \begin{rSubsection}{Teaching assistant}{2023-24}{Computer Architectures and
      Operating Systems}{Prof. Federico Terraneo }
  \item Exercise lectures: Linux Operating Systems.
  \item Topics addressed (partial): parallel programming (processes, threads),
    task scheduler, system calls and interrupt routines, memory management, file
    systems and I/O.
  \end{rSubsection}
  \begin{rSubsection}{Teaching assistant}{2021-22;\ 22-23}{Informatica (per Aerospaziali)}{Prof.\ Gerardo Pelosi}
  \item Exercise lectures: computer science for Aerospace Engineering.
  \item Topics addressed (partial): Boolean logic and basics of C programming.
  \end{rSubsection}
\begin{rSubsection}{Teaching tutor}{2019}{Informatica (per Ambientali)}{Prof.\ Andrea Bonarini}
\item Theory lectures and laboratory exercises on the C programming language.
% https://www11.ceda.polimi.it/schedaincarico/schedaincarico/controller/scheda_pubblica/SchedaPublic.do?&evn_default=evento&c_classe=691950&polij_device_category=DESKTOP&__pj0=0&__pj1=ee13b0b6c0f983f39edcd5af2ad25bd3
\end{rSubsection}
\begin{rSubsection}{Teaching tutor}{2018}{Computer Architectures and Operating Systems}{Prof.ssa Anna Maria Antola}
\item Theory lectures and lab exercises regarding both the architectures of
  modern computers (ranging from the assembly language to the logic gates) and
  the structure of an operating system (including the theory of parallel
  programming and threads management in Linux)
% https://www11.ceda.polimi.it/schedaincarico/schedaincarico/controller/scheda_pubblica/SchedaPublic.do?&evn_default=evento&c_classe=667442&polij_device_category=DESKTOP&__pj0=0&__pj1=f7cded5163ff00a70c340961c9ce685c
\end{rSubsection}
\end{rSection}
\clearpage
\begin{rSection}{List of publications}
  \begin{tSubPublications}{Journals}{J}
  \item \fullcite{perriello2023ImprovingEfficiencyQuantum}
  \end{tSubPublications}
  \begin{tSubPublications}{Conferences}{C}
  \item \fullcite{DBLP:conf/qce/PerrielloBP21}
  \item \fullcite{DBLP:conf/securecomm/PerrielloBP21}
  \end{tSubPublications}
\end{rSection}
\begin{rSection}{Scientific Community roles}
  \begin{rSubsection}{Reviewer}{}{}{}
  \item \fullcite{DBLP:conf/icissp/2023}
  \item \fullcite{DBLP:conf/acisp/2023}
  \item \fullcite{DBLP:conf/iccad/2021}
  \end{rSubsection}
\end{rSection}
\begin{rSection}{Additional scientific activities}
  \begin{itemize}
  \item[2021] Poster presenter at \emph{International Summer School on Advanced
      Computer Architecture and Compilation for High-performance Embedded
      Systems } with title \emph{A Quantum Circuit to Speed-up the Cryptanalysis
      of Code-based Cryptosystems}
  \end{itemize}
\end{rSection}
\begin{rSection}{Awards and Recognition}
  \begin{itemize}
  \item[2021] Grant winner for \emph{International Summer School on Advanced
      Computer Architecture and Compilation for High-performance Embedded
      Systems}
  \end{itemize}
\end{rSection}
\vfill
\ifpublic{%
    \begin{itemize}
\item
I authorize the processing of data pursuant to GDPR 2016/679 of 27 April 2016 (European Regulation concerning the protection of individuals with regard to the processing of personal data).
\item
    I authorize the publication of the Curriculum Vitae on the website of the Politecnico di Milano (Section Transparent Administration) in compliance with Legislative Decree no. 33 of March 14, 2013 and any subsequent amendments and additions
\end{itemize}
    \begin{itemize}
\item Autorizzo al trattamento dati ai sensi del GDPR 2016/679 del 27 aprile 2016 (Regolamento Europeo relativo alla protezione delle persone fisiche per quanto riguarda il trattamento dei dati personali).
\item Autorizzo la pubblicazione del Curriculum Vitae sul sito istituzionale del Politecnico di Milano (sez. Amministrazione Trasparente) in ottemperanza al D. Lgs n. 33 del 14 marzo 2013 (e s.m.i.).
\end{itemize}
  }
\fi
\end{document}

% Add reviewer ICISSP 2023
% Add reviewer ICCAD 2021